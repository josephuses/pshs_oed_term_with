\documentclass[]{article}
\usepackage{lmodern}
\usepackage{amssymb,amsmath}
\usepackage{ifxetex,ifluatex}
\usepackage{fixltx2e} % provides \textsubscript
\ifnum 0\ifxetex 1\fi\ifluatex 1\fi=0 % if pdftex
  \usepackage[T1]{fontenc}
  \usepackage[utf8]{inputenc}
\else % if luatex or xelatex
  \ifxetex
    \usepackage{mathspec}
  \else
    \usepackage{fontspec}
  \fi
  \defaultfontfeatures{Ligatures=TeX,Scale=MatchLowercase}
\fi
% use upquote if available, for straight quotes in verbatim environments
\IfFileExists{upquote.sty}{\usepackage{upquote}}{}
% use microtype if available
\IfFileExists{microtype.sty}{%
\usepackage{microtype}
\UseMicrotypeSet[protrusion]{basicmath} % disable protrusion for tt fonts
}{}
\usepackage[margin=1in]{geometry}
\usepackage{hyperref}
\hypersetup{unicode=true,
            pdftitle={How Many Scholarship Slots are too Many in the PSHS Campuses?},
            pdfauthor={Joseph S. Tabadero, Jr.},
            pdfborder={0 0 0},
            breaklinks=true}
\urlstyle{same}  % don't use monospace font for urls
\usepackage{longtable,booktabs}
\usepackage{graphicx,grffile}
\makeatletter
\def\maxwidth{\ifdim\Gin@nat@width>\linewidth\linewidth\else\Gin@nat@width\fi}
\def\maxheight{\ifdim\Gin@nat@height>\textheight\textheight\else\Gin@nat@height\fi}
\makeatother
% Scale images if necessary, so that they will not overflow the page
% margins by default, and it is still possible to overwrite the defaults
% using explicit options in \includegraphics[width, height, ...]{}
\setkeys{Gin}{width=\maxwidth,height=\maxheight,keepaspectratio}
\IfFileExists{parskip.sty}{%
\usepackage{parskip}
}{% else
\setlength{\parindent}{0pt}
\setlength{\parskip}{6pt plus 2pt minus 1pt}
}
\setlength{\emergencystretch}{3em}  % prevent overfull lines
\providecommand{\tightlist}{%
  \setlength{\itemsep}{0pt}\setlength{\parskip}{0pt}}
\setcounter{secnumdepth}{0}
% Redefines (sub)paragraphs to behave more like sections
\ifx\paragraph\undefined\else
\let\oldparagraph\paragraph
\renewcommand{\paragraph}[1]{\oldparagraph{#1}\mbox{}}
\fi
\ifx\subparagraph\undefined\else
\let\oldsubparagraph\subparagraph
\renewcommand{\subparagraph}[1]{\oldsubparagraph{#1}\mbox{}}
\fi

%%% Use protect on footnotes to avoid problems with footnotes in titles
\let\rmarkdownfootnote\footnote%
\def\footnote{\protect\rmarkdownfootnote}

%%% Change title format to be more compact
\usepackage{titling}

% Create subtitle command for use in maketitle
\newcommand{\subtitle}[1]{
  \posttitle{
    \begin{center}\large#1\end{center}
    }
}

\setlength{\droptitle}{-2em}

  \title{How Many Scholarship Slots are too Many in the PSHS Campuses?}
    \pretitle{\vspace{\droptitle}\centering\huge}
  \posttitle{\par}
  \subtitle{Counting Top Passers, and Comparing the Relative Performances Against
Principal and Alternate Qualifiers}
  \author{Joseph S. Tabadero,
Jr.\thanks{Special Science Teacher IV, Philippine Science High School-CAR Campus}}
    \preauthor{\centering\large\emph}
  \postauthor{\par}
      \predate{\centering\large\emph}
  \postdate{\par}
    \date{9/6/2018}


\begin{document}
\maketitle

{
\setcounter{tocdepth}{3}
\tableofcontents
}
\hypertarget{executive-summary}{%
\subsection{Executive Summary}\label{executive-summary}}

\begin{itemize}
\tightlist
\item
  At least three indicators can be set to determine the capability of a
  campus to offer more scholarship slots: (1) infrastructure; (2) the
  depth of the pool of qualified applicants by campus of choice; and (3)
  the capability of instructional services, by campus, to bring every
  accepted scholar up to PSHS standards. In this paper, we are going to
  focus on indicators 2 and 3.
\item
  To investigate indicator 2, we shall look at the top 1590 passers
  nationwide to determine the number of scholars in each campus who
  belonged to the top 1590 and the number of principal qualifiers in
  each campus who are not in the top 1590, nationwide. The number 1590
  is chosen because that is the sum of slots available if we assume 240
  slots for the Main Campus and 90 slots for each of the 15 regional
  campuses.
\item
  An \emph{NCE passer} is defined as any NCE applicant whose score in
  each subject is above the subject mean score for the applicable NCE
  year. A \emph{principal qualifier} is an NCE passer who either belongs
  to the top 240 NCE passers nationwide or to the top 90 NCE passers for
  his or her campus of first choice. We note that a principal qualifier
  may not belong to the top 1590 NCE passers, nationwide.
\item
  Additionally, we shall look at the top 3180 passers nationwide.
  Ideally, the alternate qualifiers should belong to the top 3180
  passers nationwide.
\item
  Based on the count of top 1590 passers who signified interest for each
  campus, \textbf{Main Campus, CLC, and CBZRC} have passers that almost
  double the available scholarship slots, throughout the years.
  Meanwhile, SMC and WVC can offer 30-60 more scholarship slots.
\item
  Regarding alternate qualifiers, based on the top 3180 passers
  nationwide (to diminish burden on instruction):

  \begin{itemize}
  \tightlist
  \item
    BRC can choose from among 50-150 alternates.
  \item
    CARC can choose from among 20-60 alternates.
  \item
    CBZRC can choose from among 100-300 alternates.
  \item
    CLC can choose from among 100-400 alternates.
  \item
    CVC can choose from among 30-90 alternates.
  \item
    CVISC can choose from among 30-60 alternates.
  \item
    EVC can choose from among 39-100 alternates.
  \item
    IRC can choose from among 30-60 alternates.
  \item
    Main Campus can choose from about 500 alternates.
  \item
    SMC can choose from about 150 alternates.
  \item
    WVC can choose from 150-200 alternates.
  \end{itemize}
\item
  On the other hand, CRC, MRC, SRC, and ZRC may opt not to enrol some
  more alternates in order to lessen the burden of instruction. These
  campuses should campaign for more applicants.
\item
  In terms of the relative performances in the Readiness Examination of
  the principal and alternative qualifiers, there is no appreciable
  difference, with the principal qualifiers only having a slight edge in
  math, physics and statistics. This may mean two things:

  \begin{itemize}
  \tightlist
  \item
    The alternative qualifiers can catch up with the principal
    qualifiers in a level learning field.
  \item
    The instructional services in each campus is capable in bringing
    every scholar up to par.
  \end{itemize}
\end{itemize}

\hypertarget{philippine-science-high-school-everywhere}{%
\subsection{Philippine Science High School
Everywhere}\label{philippine-science-high-school-everywhere}}

With the establishment of the MIMAROPA campus, the Philippine Science
High School is now one region away from establishing a campus in every
region in the country---as mandated by Republic Act 8496 (otherwise
known as ``Philippine Science High School (PSHS) System Act of 1997'')
as amended by Republich Act 9036. This mandate empowers the PSHS
System's Board of Trustees to ``develop policies for the expansion of
enrollment in campuses under the PSHS System'' (Section 7 (j)). However,
up until CALABARZON Campus increased the number of scholarship slots,
there has yet to be a system by which the BOT can determine the
readiness of a campus to expand expand enrollment.

We look at three factors affecting the capability of a campus to expand
enrollment: (1) available infrastructure and budget; (2) the depth of
the pool of qualified applicants by campus of choice; and (3) the
capability of instructional services to bring every accepted scholar up
to PSHS standards. The first factor is easily ascertained by inspection.
In this paper, we investigate factors 2 and 3 and propose indicators to
for each.

\hypertarget{counting-qualified-applicants}{%
\subsection{Counting Qualified
Applicants}\label{counting-qualified-applicants}}

While the PSHS System is mandated to provide equal opportunity for each
region to have an access to a PSHS campus, RA 8496 also requires
standards for it to follow by ``offering scholarships to deserving
students'' (Title I, Sec. 4). This is done through a rigorous selection
process starting with the National Competitive Examination (NCE).

The NCE is composed of 4 tests---verbal, mathematics, science, and
abstract reasoning. An \emph{NCE passer} is defined as any NCE applicant
whose score in each subject is above the subject mean score for the
applicable NCE year. A \emph{principal qualifier} is an NCE passer who
either belongs to the top 240 NCE passers nationwide or to the top 90
NCE passers for his or her campus of first choice.

A natural choice of indicator for the second factor is the number of
passers each campus of first choice has in the top 1590 passing
examinees nationwide. Here, we note that a principal qualifier in a
campus might not be in the top 1590 passers. We arrive at the number
1590 by assuming 240 scholarship slots for Main Campus and 90
scholarship slots for each of the 15 existing regional campuses. Our
assumption is that if the quality of applicants are the same throughout
the regions of the country, then the proportion of NCE passers in the
top 1590 from each region should be the same. This indicator can then
show us: (i) how many NCE passers who belonged to the top 1590 did not
qualify as a principal qualifier; and, (ii) how many principal
qualifiers do not belong to the top 1590 NCE passers. Counting the
number of the top 1590 NCE passers in each campus gives us an idea of
whether or not we are giving scholarship slots to the most deserving
applicants, while maintaining equal opportunity for applicants from the
different regions to avail of a scholarship.

The top 3180 NCE passers, on the other hand, can give us an idea on how
deep in the list of alternate qualifiers the PSHS can go and still
ensure that the quality of scholars is not compromised. Again, we can
count the number of passers for each campus of choice who belong to the
top 3180 NCE passers.

\hypertarget{how-good-is-the-quality-of-instruction-in-pshs}{%
\subsection{How good is the quality of instruction in
PSHS?}\label{how-good-is-the-quality-of-instruction-in-pshs}}

A critical factor in expanding enrollment is the availability of
qualified teachers for each of the regional campuses. The members of the
faculty of PSHS are usually degree holders in specialized fields. A
measure of the quality of instruction given by teachers is the Readiness
Examination (RE) score. The RE is given to Grades 8 and 10 in order to
determine the readiness of scholars to pursue more advanced studies in
line with the included subjects: integrated science, mathematics, social
science, english, statistics, filipino.

It can been shown that economic status correlates positively with NCE
performance. Instruction should be able to improve the readiness of
every scholar---principal qualifier or otherwise. Instruction in PSHS
should bridge the gap brought by socio-economic background and level the
playing field for each scholar. In such an environment, there should
only be a few extremely exceptional scholars---all else should perform
almost equally well.

A comparison of the distribution of RE scores of principal and alternate
qualifiers is therefore required.

\hypertarget{principal-qualifiers}{%
\subsection{Principal Qualifiers}\label{principal-qualifiers}}

\begin{table}

\caption{\label{tab:unnamed-chunk-5}Number of students belonging to the top 1590 examinees by signified campus of first choice, $N_1$, and the number of students who did not make it as a principal qualifier in their choice campus (positive) or the number of students who did not make it to the top 1590 nationally but made it as principal qualifiers (negative) by region, $N_2$ ($N_1: N_2$).}
\centering
\begin{tabular}[t]{llllllll}
\toprule
Campus & 2011 & 2012 & 2013 & 2014 & 2015 & 2016 & 2017\\
\midrule
BRC & 44: -46 & 54: -36 & 98: 8 & 103: 13 & 98: 8 & 85: -5 & 77: -13\\
CARC & 24: -66 & 38: -52 & 63: -27 & 71: -19 & 44: -46 & 70: -20 & 60: -30\\
CBZRC & 0: -90 & 0: -90 & 0: -90 & 155: 65 & 152: 62 & 228: 138 & 223: 133\\
CLC & 58: -32 & 44: -46 & 234: 144 & 237: 147 & 172: 82 & 216: 126 & 192: 102\\
CMC & 38: -52 & 39: -51 & 48: -42 & 73: -17 & 66: -24 & 54: -36 & 48: -42\\
\addlinespace
CRC & 0: -90 & 0: -90 & 29: -61 & 46: -44 & 39: -51 & 68: -22 & 47: -43\\
CVC & 33: -57 & 24: -66 & 84: -6 & 99: 9 & 74: -16 & 62: -28 & 58: -32\\
CVISC & 42: -48 & 59: -31 & 72: -18 & 68: -22 & 84: -6 & 86: -4 & 78: -12\\
EVC & 71: -19 & 50: -40 & 68: -22 & 72: -18 & 81: -9 & 89: -1 & 73: -17\\
IRC & 39: -51 & 35: -55 & 77: -13 & 76: -14 & 71: -19 & 55: -35 & 34: -56\\
\addlinespace
MC & 1165: 925 & 1102: 862 & 716: 476 & 599: 359 & 457: 217 & 637: 397 & 513: 273\\
MRC & 0: -90 & 0: -90 & 0: -90 & 0: -90 & 2: -88 & 0: -90 & 5: -85\\
SMC & 108: 18 & 118: 28 & 116: 26 & 156: 66 & 138: 48 & 115: 25 & 119: 29\\
SRC & 0: -90 & 10: -80 & 24: -66 & 28: -62 & 32: -58 & 43: -47 & 36: -54\\
WVC & 107: 17 & 111: 21 & 141: 51 & 150: 60 & 147: 57 & 128: 38 & 163: 73\\
ZRC & 0: -90 & 0: -90 & 0: -90 & 0: -90 & 5: -85 & 14: -76 & 20: -70\\
\bottomrule
\end{tabular}
\end{table}

\includegraphics{principal_and_alternate_qualifiers_files/figure-latex/unnamed-chunk-6-1.pdf}

\hypertarget{alternates}{%
\subsection{Alternates}\label{alternates}}

\begin{longtable}[]{@{}llllllll@{}}
\toprule
Campus & 2011 & 2012 & 2013 & 2014 & 2015 & 2016 & 2017\tabularnewline
\midrule
\endhead
BRC & 95: 5 & 130: 40 & 222: 132 & 206: 116 & 207: 117 & 180: 90 & 175:
85\tabularnewline
CARC & 64: -26 & 98: 8 & 132: 42 & 145: 55 & 98: 8 & 150: 60 & 128:
38\tabularnewline
CBZRC & 0: -90 & 0: -90 & 0: -90 & 313: 223 & 295: 205 & 409: 319 & 396:
306\tabularnewline
CLC & 117: 27 & 87: -3 & 472: 382 & 442: 352 & 336: 246 & 411: 321 &
329: 239\tabularnewline
CMC & 91: 1 & 97: 7 & 115: 25 & 159: 69 & 133: 43 & 144: 54 & 112:
22\tabularnewline
CRC & 0: -90 & 0: -90 & 62: -28 & 84: -6 & 71: -19 & 136: 46 & 97:
7\tabularnewline
CVC & 63: -27 & 79: -11 & 171: 81 & 179: 89 & 116: 26 & 154: 64 & 124:
34\tabularnewline
CVISC & 119: 29 & 119: 29 & 161: 71 & 143: 53 & 158: 68 & 149: 59 & 146:
56\tabularnewline
EVC & 144: 54 & 130: 40 & 159: 69 & 145: 55 & 168: 78 & 209: 119 & 144:
54\tabularnewline
IRC & 79: -11 & 63: -27 & 153: 63 & 121: 31 & 126: 36 & 111: 21 & 61:
-29\tabularnewline
MC & 2094: 1854 & 1997: 1757 & 1314: 1074 & 1076: 836 & 915: 675 & 1160:
920 & 969: 729\tabularnewline
MRC & 0: -90 & 0: -90 & 0: -90 & 0: -90 & 4: -86 & 2: -88 & 8:
-82\tabularnewline
SMC & 236: 146 & 243: 153 & 258: 168 & 289: 199 & 283: 193 & 240: 150 &
238: 148\tabularnewline
SRC & 0: -90 & 30: -60 & 56: -34 & 64: -26 & 67: -23 & 91: 1 & 69:
-21\tabularnewline
WVC & 204: 114 & 240: 150 & 277: 187 & 271: 181 & 239: 149 & 290: 200 &
301: 211\tabularnewline
ZRC & 0: -90 & 0: -90 & 0: -90 & 0: -90 & 16: -74 & 34: -56 & 39:
-51\tabularnewline
\bottomrule
\end{longtable}

\includegraphics{principal_and_alternate_qualifiers_files/figure-latex/unnamed-chunk-11-1.pdf}

\includegraphics{principal_and_alternate_qualifiers_files/figure-latex/unnamed-chunk-15-1.pdf}

\includegraphics{principal_and_alternate_qualifiers_files/figure-latex/unnamed-chunk-17-1.pdf}

\hypertarget{bicol-region-campus}{%
\subsubsection{Bicol Region Campus}\label{bicol-region-campus}}

\includegraphics{principal_and_alternate_qualifiers_files/figure-latex/unnamed-chunk-18-1.pdf}

\hypertarget{car-campus}{%
\subsubsection{CAR Campus}\label{car-campus}}

\includegraphics{principal_and_alternate_qualifiers_files/figure-latex/unnamed-chunk-19-1.pdf}

\hypertarget{cbzrc-campus}{%
\subsubsection{CBZRC Campus}\label{cbzrc-campus}}

\includegraphics{principal_and_alternate_qualifiers_files/figure-latex/unnamed-chunk-20-1.pdf}

\hypertarget{central-luzon-campus}{%
\subsubsection{Central Luzon Campus}\label{central-luzon-campus}}

\includegraphics{principal_and_alternate_qualifiers_files/figure-latex/unnamed-chunk-21-1.pdf}

\hypertarget{central-mindanao-campus}{%
\subsubsection{Central Mindanao Campus}\label{central-mindanao-campus}}

\includegraphics{principal_and_alternate_qualifiers_files/figure-latex/unnamed-chunk-22-1.pdf}

\hypertarget{caraga-region-campus}{%
\subsubsection{Caraga Region Campus}\label{caraga-region-campus}}

\includegraphics{principal_and_alternate_qualifiers_files/figure-latex/unnamed-chunk-23-1.pdf}

\hypertarget{cagayan-valley-campus}{%
\subsubsection{Cagayan Valley Campus}\label{cagayan-valley-campus}}

\includegraphics{principal_and_alternate_qualifiers_files/figure-latex/unnamed-chunk-24-1.pdf}

\hypertarget{central-visayas-campus}{%
\subsubsection{Central Visayas Campus}\label{central-visayas-campus}}

\includegraphics{principal_and_alternate_qualifiers_files/figure-latex/unnamed-chunk-25-1.pdf}

\hypertarget{eastern-visayas-campus}{%
\subsubsection{Eastern Visayas Campus}\label{eastern-visayas-campus}}

\includegraphics{principal_and_alternate_qualifiers_files/figure-latex/unnamed-chunk-26-1.pdf}

\hypertarget{ilocos-region-campus}{%
\subsubsection{Ilocos Region Campus}\label{ilocos-region-campus}}

\includegraphics{principal_and_alternate_qualifiers_files/figure-latex/unnamed-chunk-27-1.pdf}

\hypertarget{main-campus}{%
\subsubsection{Main Campus}\label{main-campus}}

\includegraphics{principal_and_alternate_qualifiers_files/figure-latex/unnamed-chunk-28-1.pdf}

\hypertarget{mimaropa-region-campus}{%
\subsubsection{MIMAROPA Region Campus}\label{mimaropa-region-campus}}

\includegraphics{principal_and_alternate_qualifiers_files/figure-latex/unnamed-chunk-29-1.pdf}

\hypertarget{southern-mindanao-campus}{%
\subsubsection{Southern Mindanao
Campus}\label{southern-mindanao-campus}}

\includegraphics{principal_and_alternate_qualifiers_files/figure-latex/unnamed-chunk-30-1.pdf}

\hypertarget{src-campus}{%
\subsubsection{SRC Campus}\label{src-campus}}

\includegraphics{principal_and_alternate_qualifiers_files/figure-latex/unnamed-chunk-31-1.pdf}

\hypertarget{western-visayas-campus}{%
\subsubsection{Western Visayas Campus}\label{western-visayas-campus}}

\includegraphics{principal_and_alternate_qualifiers_files/figure-latex/unnamed-chunk-32-1.pdf}

\hypertarget{car-campus-1}{%
\subsubsection{CAR Campus}\label{car-campus-1}}

\includegraphics{principal_and_alternate_qualifiers_files/figure-latex/unnamed-chunk-33-1.pdf}

\hypertarget{pshs-system}{%
\subsection{PSHS System}\label{pshs-system}}

\includegraphics{principal_and_alternate_qualifiers_files/figure-latex/unnamed-chunk-34-1.pdf}


\end{document}
